\documentclass[11pt,a4paper]{article}
\usepackage[spanish,es-nodecimaldot]{babel}	% Utilizar español
\usepackage[utf8]{inputenc}					% Caracteres UTF-8
\usepackage{graphicx}						% Imagenes
\usepackage[hidelinks]{hyperref}			% Poner enlaces sin marcarlos en rojo
\usepackage{fancyhdr}						% Modificar encabezados y pies de pagina
\usepackage{float}							% Insertar figuras
\usepackage[textwidth=390pt]{geometry}		% Anchura de la pagina
\usepackage[nottoc]{tocbibind}				% Referencias (no incluir num pagina indice en Indice)
\usepackage{enumitem}						% Permitir enumerate con distintos simbolos
\usepackage[T1]{fontenc}					% Usar textsc en sections
\usepackage{amsmath}						% Símbolos matemáticos
\usepackage{listings}
\usepackage{color}

 
\definecolor{codegreen}{rgb}{0,0.6,0}
\definecolor{codegray}{rgb}{0.5,0.5,0.5}
\definecolor{codepurple}{rgb}{0.58,0,0.82}
\definecolor{backcolour}{rgb}{0.99,0.99,0.99}
 
\lstdefinestyle{mystyle}{
    backgroundcolor=\color{backcolour},   
    commentstyle=\color{codegreen},
    keywordstyle=\color{magenta},
    numberstyle=\tiny\color{codegray},
    stringstyle=\color{codepurple},
    basicstyle=\footnotesize,
    breakatwhitespace=false,         
    breaklines=true,                 
    captionpos=b,                    
    keepspaces=true,                 
    numbers=left,                    
    numbersep=5pt,                  
    showspaces=false,                
    showstringspaces=false,
    showtabs=false,                  
    tabsize=2
}
 
\lstset{style=mystyle, language=Python}

% Comando para poner el nombre de la asignatura
\newcommand{\asignatura}{Inteligencia de Negocio}
\newcommand{\autor}{José María Sánchez Guerrero}
\newcommand{\titulo}{Práctica 1}
\newcommand{\subtitulo}{Resolución de problemas de clasificación y análisis experimental.}

% Configuracion de encabezados y pies de pagina
\pagestyle{fancy}
\lhead{\autor{}}
\rhead{\asignatura{}}
\lfoot{Grado en Ingeniería Informática}
\cfoot{}
\rfoot{\thepage}
\renewcommand{\headrulewidth}{0.4pt}		% Linea cabeza de pagina
\renewcommand{\footrulewidth}{0.4pt}		% Linea pie de pagina

\begin{document}
\pagenumbering{gobble}

% Pagina de titulo
\begin{titlepage}

\begin{minipage}{\textwidth}

\centering

\includegraphics[scale=0.5]{img/ugr.png}\\

\textsc{\Large \asignatura{}\\[0.2cm]}
\textsc{GRADO EN INGENIERÍA INFORMÁTICA}\\[1cm]

\noindent\rule[-1ex]{\textwidth}{1pt}\\[1.5ex]
\textsc{{\Huge \titulo\\[0.5ex]}}
\textsc{{\Large \subtitulo\\}}
\noindent\rule[-1ex]{\textwidth}{2pt}\\[3.5ex]

\end{minipage}

\vspace{0.5cm}

\begin{minipage}{\textwidth}

\centering

\textbf{Autor}\\ {\autor{}}\\[2.5ex]
\textbf{Rama}\\ {Computación y Sistemas Inteligentes}\\[2.5ex]
\vspace{0.3cm}

\includegraphics[scale=0.3]{img/etsiit.jpeg}

\vspace{0.3cm}
\textsc{Escuela Técnica Superior de Ingenierías Informática y de Telecomunicación}\\
\vspace{1cm}
\textsc{Curso 2020-2021}
\end{minipage}
\end{titlepage}

\pagenumbering{arabic}
\tableofcontents
\thispagestyle{empty}				% No usar estilo en la pagina de indice

\newpage

\setlength{\parskip}{1em}



\section{Introducción}

En este trabajo vamos a analizar el comportamiento de distintos algoritmos de clasificación en el problema propuesto. Disponemos
de un dataset, llamado \textit{''Mammographic Mass dataset''}, en el cual se desea predecir el tipo de tumor (benigno o maligno)
en una serie de mamografías realizadas para un estudio sobre el cáncer de mama. Este estudio lo vamos a realizar gracias a los
siguientes atributos proporcionados en el dataset:

\begin{itemize}
	\item \textbf{BI-RADS.} Este parámetro representa un control de calidad de las mamografías. Consta de 7 categorías distintas,
		  en las que, cuanto más alto sea el valor, hay una mayor probabilidad de que sea maligno.

	\item \textbf{Edad} del paciente.
	
	\item \textbf{Forma de la masa.} Dependiendo de como sea la masa anormal detectada, se clasifica como \textbf{R}edondeada,
		  \textbf{O}valada, \textbf{L}obulada, \textbf{I}rregular ó \textbf{N}o definida.

	\item \textbf{Margen de masa.} Circumscribed = 1, microlobulated = 2, obscured = 3, ill-defined = 4, spiculated = 5 (nominal).
	
	\item \textbf{Densidad de la masa.} Valores entre 1 y 4, siendo 1 la más alta y 4 contenido graso (no tumoral).
	
	\item \textbf{Severidad.} Es el atributo que se desea predecir, es decir, si es un tumor benigno o maligno.	

\end{itemize}

En el dataset hay datos de 961 pacientes, sin embargo, nos gustaría dejar un porcentaje para validar el modelo y así ver cómo va
entrenando los datos.



\section{Procesado de datos}


\section{Configuración de algoritmos}


\section{Resultados obtenidos}


\section{Interpretación de resultados}


\section{Bibliografía}


\end{document}