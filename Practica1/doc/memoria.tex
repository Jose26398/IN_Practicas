\documentclass[11pt,a4paper]{article}
\usepackage[spanish,es-nodecimaldot]{babel}	% Utilizar español
\usepackage[utf8]{inputenc}					% Caracteres UTF-8
\usepackage{graphicx}						% Imagenes
\usepackage[hidelinks]{hyperref}			% Poner enlaces sin marcarlos en rojo
\usepackage{fancyhdr}						% Modificar encabezados y pies de pagina
\usepackage{float}							% Insertar figuras
\usepackage[textwidth=390pt]{geometry}		% Anchura de la pagina
\usepackage[nottoc]{tocbibind}				% Referencias (no incluir num pagina indice en Indice)
\usepackage{enumitem}						% Permitir enumerate con distintos simbolos
\usepackage[T1]{fontenc}					% Usar textsc en sections
\usepackage{amsmath}						% Símbolos matemáticos
\usepackage{listings}
\usepackage{color}

 
\definecolor{codegreen}{rgb}{0,0.6,0}
\definecolor{codegray}{rgb}{0.5,0.5,0.5}
\definecolor{codepurple}{rgb}{0.58,0,0.82}
\definecolor{backcolour}{rgb}{0.99,0.99,0.99}
 
\lstdefinestyle{mystyle}{
    backgroundcolor=\color{backcolour},   
    commentstyle=\color{codegreen},
    keywordstyle=\color{magenta},
    numberstyle=\tiny\color{codegray},
    stringstyle=\color{codepurple},
    basicstyle=\footnotesize,
    breakatwhitespace=false,         
    breaklines=true,                 
    captionpos=b,                    
    keepspaces=true,                 
    numbers=left,                    
    numbersep=5pt,                  
    showspaces=false,                
    showstringspaces=false,
    showtabs=false,                  
    tabsize=2
}
 
\lstset{style=mystyle, language=Python}

% Comando para poner el nombre de la asignatura
\newcommand{\asignatura}{Inteligencia de Negocio}
\newcommand{\autor}{José María Sánchez Guerrero}
\newcommand{\titulo}{Práctica 1}
\newcommand{\subtitulo}{Resolución de problemas de clasificación y análisis experimental.}

% Configuracion de encabezados y pies de pagina
\pagestyle{fancy}
\lhead{\autor{}}
\rhead{\asignatura{}}
\lfoot{Grado en Ingeniería Informática}
\cfoot{}
\rfoot{\thepage}
\renewcommand{\headrulewidth}{0.4pt}		% Linea cabeza de pagina
\renewcommand{\footrulewidth}{0.4pt}		% Linea pie de pagina

\begin{document}
\pagenumbering{gobble}

% Pagina de titulo
\begin{titlepage}

\begin{minipage}{\textwidth}

\centering

\includegraphics[scale=0.5]{img/ugr.png}\\

\textsc{\Large \asignatura{}\\[0.2cm]}
\textsc{GRADO EN INGENIERÍA INFORMÁTICA}\\[1cm]

\noindent\rule[-1ex]{\textwidth}{1pt}\\[1.5ex]
\textsc{{\Huge \titulo\\[0.5ex]}}
\textsc{{\Large \subtitulo\\}}
\noindent\rule[-1ex]{\textwidth}{2pt}\\[3.5ex]

\end{minipage}

\vspace{0.5cm}

\begin{minipage}{\textwidth}

\centering

\textbf{Autor}\\ {\autor{}}\\[2.5ex]
\textbf{Rama}\\ {Computación y Sistemas Inteligentes}\\[2.5ex]
\vspace{0.3cm}

\includegraphics[scale=0.3]{img/etsiit.jpeg}

\vspace{0.3cm}
\textsc{Escuela Técnica Superior de Ingenierías Informática y de Telecomunicación}\\
\vspace{1cm}
\textsc{Curso 2020-2021}
\end{minipage}
\end{titlepage}

\pagenumbering{arabic}
\tableofcontents
\thispagestyle{empty}				% No usar estilo en la pagina de indice

\newpage

\setlength{\parskip}{1em}



\section{Introducción}

En este trabajo vamos a analizar el comportamiento de distintos algoritmos de clasificación en el problema propuesto. Disponemos
de un dataset, llamado \textit{''Mammographic Mass dataset''}, en el cual se desea predecir el tipo de tumor (benigno o maligno)
en una serie de mamografías realizadas para un estudio sobre el cáncer de mama. Este estudio lo vamos a realizar gracias a los
siguientes atributos proporcionados en el dataset:

\begin{itemize}
	\item \textbf{BI-RADS.} Este parámetro representa un control de calidad de las mamografías. Consta de 7 categorías distintas,
		  en las que, cuanto más alto sea el valor, hay una mayor probabilidad de que sea maligno.

	\item \textbf{Edad} del paciente.
	
	\item \textbf{Forma de la masa.} Dependiendo de como sea la masa anormal detectada, se clasifica como \textbf{R}edondeada,
		  \textbf{O}valada, \textbf{L}obulada, \textbf{I}rregular ó \textbf{N}o definida.

	\item \textbf{Margen de masa.} Circumscribed = 1, microlobulated = 2, obscured = 3, ill-defined = 4, spiculated = 5 (nominal).
	
	\item \textbf{Densidad de la masa.} Valores entre 1 y 4, siendo 1 la más alta y 4 contenido graso (no tumoral).
	
	\item \textbf{Severidad.} Es el atributo que se desea predecir, es decir, si es un tumor benigno o maligno.	

\end{itemize}

En el dataset hay datos de 961 pacientes, sin embargo, nos gustaría dejar un porcentaje para validar el modelo y así ver cómo va
entrenando los datos. Posteriormente, se explicará cómo se ha determinado qué datos son los de entrenamiento y cuáles son los de
test.

\newpage

\section{Procesado de datos}

Lo primero que tenemos que hacer es mostrar varios de los datos que tenemos y analizarlos. En mi caso vamos a sacar las 5 primeras
filas:

\begin{table}[H]
    \centering
    \resizebox{\textwidth}{!}{%
    \begin{tabular}{|c|c|c|c|c|c|c|}
    \hline
    \textbf{} & \textbf{BI-RADS} & \textbf{Age} & \textbf{Shape} & \textbf{Margin} & \textbf{Density} & \textbf{Severity} \\ \hline
    \textbf{0} & 5.0 & 67.0 & L & 5.0 & 3.0 & maligno \\ \hline
    \textbf{1} & 4.0 & 43.0 & R & 1.0 & NaN & maligno \\ \hline
    \textbf{2} & 5.0 & 58.0 & I & 5.0 & 3.0 & maligno \\ \hline
    \textbf{3} & 4.0 & 28.0 & R & 1.0 & 3.0 & benigno \\ \hline
    \textbf{4} & 5.0 & 74.0 & R & 5.0 & NaN & maligno \\ \hline
    \end{tabular}%
    }
\end{table}

Podemos observar que tenemos tanto datos numéricos, como datos categóricos, como ya comentamos en la introducción. También podemos
observar que tenemos varias celdas con datos erróneos o perdidos (representados con el valor $NaN$), por lo que será importante
procesarlos para que nuestros algoritmos funcionen correctamente. Primero veamos qué cantidad de estos datos nulos tenemos:

\begin{table}[H]
    \centering
    \begin{tabular}{|l|l|}
    \hline
    \multicolumn{1}{|c|}{\textbf{BI-RADS}} & 2 \\ \hline
    \textbf{Age} & 5 \\ \hline
    \textbf{Shape} & 0 \\ \hline
    \textbf{Margin} & 48 \\ \hline
    \textbf{Density} & 76 \\ \hline
    \textbf{Severity} & 0 \\ \hline
    \end{tabular}
\end{table}

Son una cantidad bastante alta de datos, en comparación con la cantidad de datos totales que tenemos. Por lo tanto, eliminar toda
las filas que contengan uno, puede dejarnos con muy pocos datos para entrenar y validar, y que el modelo sea más débil. No obstante,
el introducir datos para reemplazar uno faltante ha de realizarse con cuidado, ya que no son datos reales.

También tenemos que tener en cuenta cuál es la distribución de estos datos antes de trabajar con ellos. Es decir, tenemos que
asegurarnos de no sesgar nuestros datos si los eliminamos. Si hay algún tipo de correlación, tendríamos que intentar completarlos
de alguna forma. Para ello, vamos a generar una gráfica para cada uno de los atributos del $dataset$, en la que mostraremos la
cantidad de datos antes y después de eliminarlos, y asi ver cómo están distribuidos.

\newpage
Los resultados son los siguientes:

\begin{figure}[H]
\centering

\begin{minipage}{0.5\textwidth}
    \centering
    \includegraphics[scale=0.35]{img/birads-distribution.png}
\end{minipage}%
\begin{minipage}{0.5\textwidth}
    \centering
    \includegraphics[scale=0.35]{img/age-distribution.png}
\end{minipage}

\end{figure}


\begin{figure}[H]
\centering

\begin{minipage}{0.5\textwidth}
    \centering
    \includegraphics[scale=0.35]{img/shape-distribution.png}
\end{minipage}%
\begin{minipage}{0.5\textwidth}
    \centering
    \includegraphics[scale=0.35]{img/margin-distribution.png}
\end{minipage}

\end{figure}


\begin{figure}[H]
\centering

\begin{minipage}{0.5\textwidth}
    \centering
    \includegraphics[scale=0.35]{img/density-distribution.png}
\end{minipage}%
\begin{minipage}{0.5\textwidth}
    \centering
    \includegraphics[scale=0.35]{img/severity-distribution.png}
\end{minipage}

\end{figure}

Podemos observar que los datos nulos están distribuidos aleatoriamente entre los atributos, por lo que podremos eliminarlos
del $dataset$ sin ningún problema (y teniendo en cuenta que tendremos menos datos para trabajar).

% PROBAR REEMPLAZANDO LOS DATOS CON ALEATORIOS
% \textit{Pandas} nos proporciona la función ''\textit{fillna()}'' para reemplazar los valores faltantes con un valor específico



\section{Configuración de algoritmos}

\subsection{K-Nearest-Neighbors (k-NN)}


\subsection{Decision Tree}


\subsection{Naive-Bayes}


\subsection{Neural Network}


\subsection{Support Vector Machine (SVM)}


\section{Resultados obtenidos}


\section{Interpretación de resultados}


\section{Bibliografía}


\end{document}